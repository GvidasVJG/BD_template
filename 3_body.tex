\newpage
\section{Dėstomoji dalis}
\textit{Tai dėstymo dalies aprašymas, kuriame įterpiama įterpiama temos apžvalga bei tiriamoji dalis. Temos apžvalgoje aprašoma idėjos brandos darbui paieška bei informacinių šaltinių apibendrinimas. Na, o tiriamosios dalies apžvalgoje aprašomas tiriamojo darbo metodas, darbo eigos aprašymas, tiriamojo darbo surinktų duomenų analizė bei tiriamojo darbo apibendrinimas.}

Pavyzdys, kaip veikia citavimas \cite{brandos_2019}. Dar vienas šaltinis \cite{eberson_nacionaline_nodate}. Kokia tais matematinė formulė – $y=f(x)=x^2$. Ir dar vienas, šiek tiek kitoks šaltinis \cite{tupinis_inovacijos_2022}. Toliau bus matematinė formulė, kuri yra itin svarbi, todėl centruota ir išskirta.

$$f_{n}(x)=x_{n}^{2}-n$$

\subsection{Temos apžvalga - literatūros analizė}
\subsubsection{Analizės dalis nr. 1}
\lipsum[4]
\begin{figure}[H]
    \centering
    \includegraphics{media/sample-img}
    \caption{Iliustracijos aprašymas}
    \label{fig:laikina_iliustracija}
\end{figure}
\subsubsection{Analizės dalis nr. 2}
\lipsum[4]
Čia galime įterpti kodą tekste \verb|print(f'Python kodas')|. Tačiau galime įterpti ir kodo bloką.
\begin{verbatim}
def suma(x, y):
  return x+y
  
x, y = 10, 20
print(suma(x, y))
\end{verbatim}
Kaip galima pastebėti, kodas nėra hilight'intas. Jei to reikia, \newline galime pasinaudoti paketu \verb|listings|:
\begin{lstlisting}[language=Python, caption=Pavyzdinis kodas]
def suma(x, y):
  return x+y
  
x, y = 10, 20
print(suma(x, y))
\end{lstlisting}

\subsection{Tyrimo apžvalga - tyrimo aprašymas ir rezultatai}
\subsubsection{Tyrimo organizavimas}
\lipsum[5]
\begin{table}[H]
    \centering
    \begin{tabular}{|c|c|}
        \hline
        \textbf{X} & \textbf{Y} \\
        \hline
        -2 & -4 \\
        -1 & -2 \\
        0 & 0 \\
        1 & 2 \\
        2 & 4 \\
        \hline
    \end{tabular}
    \caption{Lentelės aprašymas}
    \label{tab:laikina_lentele}
\end{table}
\subsubsection{Tyrimo eiga}
\lipsum[4]
\begin{figure}[H]
    \centering
    \includegraphics{media/sample-img}
    \caption{Iliustracijos aprašymas}
    \label{fig:laikina_iliustracija_2}
\end{figure}
\subsubsection{Tyrimo rezultatai}
\lipsum[5]